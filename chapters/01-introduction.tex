% Problem
In the field of process theory,
an often recurring question is that of
the behavioral similarity between processes in a transition system.
There are many different ways in which two processes can be considered
\emph{equivalent}.
In an important step towards grasping the range of equivalences,
\textcite{glabbeek1990spectrum} collected the most relevant definitions in the
so-called \emph{linear-time--branching-time spectrum}.
Being able to compute these equivalences is relevant for applications like
formal verification
or for gaining a better understanding of complex systems.

While algorithms for computing individual notions of equivalence
have long existed (e.g.~\cite{Blom2002}),
a process for deciding all equivalences of the spectrum at once
was only recently developed
by Bisping, Jansen and Nestmann~\cite{Bisping2022}.
This algorithm is subsequently refined by \textcite{bisping2023process}
where an \emph{energy game} is used for finding the equivalences.
This helps in bringing the runtime and memory usage of the algorithm down
to a more practical range,
but it is still very expensive to execute.
On larger inputs the implementation can easily run for the excess of several
minutes or even run out of memory.

% Contribution
This work aims to improve on that performance by utilizing the parallel
computing power of modern GPUs.
The idea is that large parts of the algorithm can be computed in parallel,
which, if successful, could result in a significant reduction in runtime.
To that end, the contribution of this thesis is the development
of \texttt{gpuequiv}%
\footnote{The code can be found at \url{https://github.com/Gobbel2000/gpuequiv}.
This document refers to version 1.0.0.},
a new, GPU-accelerated implementation
of the algorithm by \textcite{bisping2023process}.
It processes the most critical parts of the energy game in a highly
parallelized fashion,
with the goal of making the spectroscopy algorithm faster and more scalable.
The same implementation can also be used more generally for solving energy
games.

For the task of interfacing with the GPU,
the WebGPU API is used.
WebGPU is a new web standard, which currently is still in development,
with the goal of providing access to a system's GPU from web browsers.
By calling into WebGPU through the Rust library \texttt{wgpu} it is possible to
run the code both natively as well as from within a web site using WebAssembly.
This allows for very high platform-independence and portability.
Most of the code in \texttt{gpuequiv} is written in the
Rust programming language,
the GPU shaders are written in the WebGPU shading language WGSL\@.

\subsubsection{Related Work}

The broad structure for processing the energy game
is similar to how a lot of other graph algorithms work
by traversing the game graph in a breadth-first manner.
There is already a lot of research on running this category of graph
algorithms on a GPU~\cite{Merrill2015,Busato2018,Hijma2023}.
Similar projects exist in the context of transition systems,
for example a parallel algorithm for computing bisimilarity
by \textcite{Martens2023}
or a tool for detecting deadlocks in a transition system
by \textcite{Wijs2023},
both running on a GPU\@.
However none of them give information
regarding the entire equivalence spectrum.

\subsubsection{Structure}

The following two chapters will introduce some of the concepts required to
fully understand the implementation choices behind \texttt{gpuequiv}.
Chapter~\ref{ch:processes} starts with defining processes and equivalences,
then continues to introducing the details of
the spectroscopy algorithm (\ref{sec:spectroscopy})
and of the energy game at its core (\ref{sec:energy_games}).
Chapter~\ref{ch:gpu} focuses on another important topic for this work:
GPU programming.
This includes details on the technology stack used
and the choice of WebGPU among other GPU frameworks,
followed by a short overview on what it means
to program for a GPU (\ref{sec:gpu_model}).

The actual GPU-based implementation of the spectroscopy algorithm is explained
in Chapter~\ref{ch:implementation}.
It begins with a more high-level look at how parallelization was achieved
(\ref{sec:parallelization}),
but also includes more technical details around the data layouts
that were used to efficiently handle the data (\ref{sec:data}).
Finally, in Chapter~\ref{ch:benchmarks} we will evaluate the performance
gained by parallelization with a series of benchmarks that were
conducted both on \texttt{gpuequiv} as well as on the original implementation
by \textcite{bisping2023process}.
