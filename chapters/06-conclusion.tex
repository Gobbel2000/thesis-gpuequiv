The idea of this work was to speed up the algorithm to find all equivalences
within a transition system~\cite{bisping2023process} by running it highly
parallelized on a GPU\@.
This objective was realized in the development of the open source software
library \texttt{gpuequiv}.
It can be used to find all equivalences notions of the
linear-time--branching-time spectrum~\cite{glabbeek1990spectrum}
that equate two or more processes.
This was accomplished with a GPU-accelerated algorithm for processing energy
games that can also be utilized for energy games in general,
not just in the context of the spectroscopy algorithm.
By building on the new WebGPU specification with the implementation of
\texttt{wgpu},
the code can not only run on all major platforms,
but even inside web browsers.

We have shown in Chapter~\ref{ch:implementation} how parallelization could be
applied to the spectroscopy algorithm.
It was found that large parts of work still had to be done on the CPU,
in particular when dealing with the dynamic graph data structures.
For example generation of the energy game graph had to be done full by the CPU
for these reasons.
The size of the game graph would have made it an enticing target for
optimization,
but in the end the time spent on generating the game by the CPU does not
outweigh the more complex solving of the resulting energy game.

The most critical parts of solving the energy game could successfully be
computed in parallel on the GPU\@.
This resulted in an overall significant uplift in performance compared to the
previous sequential implementation of~\cite{bisping2023process},
as shown by the benchmarks in Table~\ref{tab:benchmarks}:
The time required to process one particularly large transition system is
reduced from nearly 4 minutes down to just 18 seconds.

Even though care was taken to store data in a highly compressed format
(see~\ref{sec:data}),
the sheer size of the game graph needed to process larger inputs still
ended up exceeding memory limits.
It was therefore not possible to meaningfully increase scalability
over the original implementation.

\subsubsection{Further work}

While we have focused on the spectrum that was used
in~\cite{bisping2023process},
there is another version of the algorithm that includes many more equivalences
by accounting for silent steps~\cite{bisping2023silent}.
This requires a larger,
more complex game graph and energy values in 8 dimensions instead of 6.
Supporting these additional equivalences would be a way of expanding the
functionality of \texttt{gpuequiv}.
And there is of course always room for further optimization.
