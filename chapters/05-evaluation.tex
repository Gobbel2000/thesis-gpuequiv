This chapter will investigate the extent at which GPU-acceleration was able to
speed up execution of the spectroscopy algorithm.
Benchmarks were conducted to compare the performance of the new,
parallelized program \texttt{gpuequiv}
with the original implementation of the same algorithm by Bisping,
created alongside~\cite{bisping2023process}.
We will further look into what was done to verify the correctness of the GPU
implementation.

\section{Benchmarks}

The performance of the implementation was assessed by running it on files from
the \enquote{Very Large Transition System Benchmark Suite} (VLTS)~\cite{vlts},
which, as the name suggests,
includes transition systems with large numbers of states.
For the benchmark, the task is to find the equivalences between \emph{all}
process pairs within a transition system.
The energy game is constructed to include all process pairs that we want to
compare,
but there are some very impactful reductions we can apply to end up with
significantly less than $|\mathsf{Proc}|^2$ starting positions.

Firstly, we can minimize the LTS by consolidating bisimilar processes.
The bisimulation of an LTS can be computed quite efficiently---certainly much
faster than the full spectroscopy algorithm---with the help of a procedure
based around partition refinement~\cite{Blom2002}.
That way we end up with a smaller transition system with just one process for
each bisimulation class of the original system.
Because bisimilarity is already the strongest notion of equivalence,
all processes within a bisimulation class must behave the same for any
equivalence comparison,
so there is no information lost by performing this reduction step.

Secondly,
we can reduce the search space by inspecting the other end of the spectrum.
If two process already aren't enabledness-equivalent,
then none of the other equivalences can hold either,
making any further spectroscopy redundant.
Comparing the enabled actions of two processes is trivially easy,
so any starting positions comparing processes with distinct action sets
can simply be discarded.

\begin{table}[htpb]
    \centering
    \caption{Benchmarks}%
    \label{tab:benchmarks}
    \small
    \begin{tabular}{@{}l
                    r@{\hskip 6pt}r
                    r@{\hskip 6pt}r
                    S[table-format=3.3]@{\hskip 6pt}
                    S[table-format=1.4]@{}
                    S[table-format=3.4]@{}}
        \toprule
        &\multicolumn{2}{c}{LTS size}
        &\multicolumn{2}{c}{Game Graph size}
        &\multicolumn{3}{c}{Runtime (s)} \\
        \cmidrule(lr){2-3} \cmidrule(lr){4-5} \cmidrule(l){6-8}
        LTS~\cite{vlts}
        &$|\mathsf{Proc}|$ &$\sim_B$
        &$|G|$ &$|E|$
        &\cite{bisping2023process} &{gpuequiv} &{Build} \\
        \midrule

        % Peterson Mutex Weak.csv 137.25ms
        \texttt{vasy\_0\_1}   &289    &9      &276        &844         &0.018   &0.0045 &0.0001  \\
        \texttt{vasy\_1\_4}   &1183   &28     &520        &1377        &0.019   &0.0053 &0.0002  \\
        \texttt{vasy\_5\_9}   &5486   &145    &1859       &3466        &0.039   &0.0051 &0.0007  \\
        \texttt{vasy\_8\_24}  &8879   &416    &59,761     &147,168     &1.347   &0.046  &0.026   \\
        \texttt{vasy\_8\_38}  &8921   &219    &7535       &21,643      &0.131   &0.0072 &0.003   \\
        \texttt{vasy\_10\_56} &10,849 &2112   &1,950,422  &6,064,252   &109.027 &4.047  &1.319   \\
        \texttt{vasy\_18\_73} &18,746 &4087   &76,010,111 &624,135,160 &{--}    &{--}   &108.872 \\
        \texttt{vasy\_25\_25} &25,217 &25,217 &0          &0           &0.217   &0.0017 &3.582   \\
        \texttt{cwi\_1\_2}    &1952   &1132   &8,504,998  &22,708,680  &228.989 &8.536  &5.197   \\
        \texttt{cwi\_3\_14}   &3996   &62     &11,094     &24,190      &0.192   &0.024  &0.003   \\
        \bottomrule
    \end{tabular}
\end{table}

\section{Correctness}
