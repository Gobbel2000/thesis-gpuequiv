\section{Parallelizing the Algorithm}

At the center of this contribution is a parallel GPU-implementation%
\footnote{The code can be found at \url{https://github.com/Gobbel2000/gpuequiv}}
of the energy game introduced in section \ref{sec:energy_games}.
Given a game graph as input, it calcu\-lates for each position the energy budgets
required for the attacker, starting at the current position, to win the game.



\begin{figure}[h]
\begin{center}
\begin{tikzpicture}[inner sep=1mm]
    \node[rectangle,draw] (start_node) {\large{$g$}};
    \node (successor1) [right=of start_node,label=above:successors] {$g_1'$};

    % Energies of successors
    \node (energy1_1) [right=of successor1] {$e_{g_1',1}$};
    \node (energy1_2) [below] at (energy1_1.south) {$e_{g_1',2}$};
    \node (dots1) [below] at (energy1_2.south) {\rvdots};
    \node (energy2_1) [below] at (dots1.south) {$e_{g_2',1}$};
    \node (energy2_2) [below] at (energy2_1.south) {$e_{g_2',2}$};
    \node (dots2) [below] at (energy2_2.south) {\rvdots};

    \node (successor2) [left=of energy2_1] {$g_2'$};
    \node (dots_successors) at (successor2 |- dots2) {\rvdots};

    % Updated energies
    \node (updated1_1) [right=of energy1_1] {$e_{g_1',1}'$};
    \node (updated1_2) [right=of energy1_2] {$e_{g_1',2}'$};
    \node (udots1) at (updated1_2 |- dots1) {\rvdots};
    \node (updated2_1) [right=of energy2_1] {$e_{g_2',1}'$};
    \node (updated2_2) [right=of energy2_2] {$e_{g_2',2}'$};
    \node (udots2) at (updated2_2 |- dots2) {\rvdots};
    % Include energies of g itself
    %\node (energyg_1) [below] at (udots2.south) {$e_{g,1}$};
    %\node (energyg_2) [below] at (energyg_1.south) {$e_{g,2}$};
    %\node (udots3) [below] at (energyg_2.south) {\rvdots};

    % Minimized energies
    \node (minimal1) [right=14mm of udots1.north] {$e_1^*$};
    \node (minimal2) [below] at (minimal1.south) {$e_2^*$};
    \node (mdots) [below] at (minimal2.south) {\rvdots};

    \draw[->] (start_node.east) -- (successor1.west);
    \draw[->] (start_node.east) -- (successor2.west);
    \draw[->] (successor1) -- (energy1_1.west);
    \draw[->] (successor1) -- (energy1_2.west);
    \draw[->] (successor1) -- (energy1_2.west |- dots1);
    \draw[->] (successor2) -- (energy2_1.west);
    \draw[->] (successor2) -- (energy2_2.west);
    \draw[->] (successor2) -- (energy2_2.west |- dots2);

    \draw[->] (energy1_1) -- node (l_update) [above=2mm] {update} (updated1_1);
    \draw[->] (energy1_2) -- (updated1_2);
    \draw[->] (energy1_2.east |- dots1) -- (updated1_2.west |- udots1);
    \draw[->] (energy2_1) -- (updated2_1);
    \draw[->] (energy2_2) -- (updated2_2);
    \draw[->] (energy2_2.east |- dots2) -- (updated2_2.west |- udots2);
    % Energies from g
    %\draw[->,shorten >= 4mm] (start_node.south) |-
    %    node [near end,below] {energies of $g$}
    %    (energyg_2.west);

    % Diagonal lines suggesting the reduction to minimal energies
    \draw[thick] (updated1_1.east |- udots2.south) -- (minimal1.west |- mdots.south);
    \draw[thick] (updated1_1.north east) -- (minimal1.north west);
    \node (l_minimize) [right=7mm of l_update] {minimize};
\end{tikzpicture}
\end{center}
\caption{Data flow for processing Attack positions}%
\label{fig:attack}
\end{figure}



\section{Data Layout}

\section{Control Flow}

\section{Limitations}
